\documentclass[a4paper,12pt]{article}
\usepackage[T1]{fontenc}
\usepackage[utf8]{inputenc}
\usepackage[brazilian]{babel}
\usepackage[top = 2cm, bottom = 2cm, left = 2.5cm, right = 2.5cm]{geometry}
\linespread{1.5}
\usepackage{setspace}
\usepackage{color}
\usepackage[dvipsnames, svqnames, xllnames]{xcolor}
\usepackage{amsmath, array, amssymb}
\usepackage{graphicx}
\usepackage{afterpage}

\begin{document}

\begin{titlepage}

\addtolength{\topmargin}{1.5cm}

\setlength{\baselineskip}{1.4\baselineskip}

\begin{center}
{\large{NOME DA UNIVERSIDADE}}

{\large{INSTITUIÇÃO ACADÊMICA OU ESCOLA OU FACULDADE}}

\end{center}
\vspace{2cm}
\begin{center}
{\Large\textbf{Ponte Reticadora}}
\end{center}

\vspace{1.5cm}

\begin{center}
{\Large{Edson Rodrigues dos Santos}}
\end{center}

\vspace{2cm}

\begin{flushright}

\begin{minipage}{10cm}

\hrulefill

"Trabalho extraido do Mundo da Elétrica e implementado com dados de experimento prático.

\hrulefill

{\textbf{Orientador: Edson Rodrigues dos Santos}}

\end{minipage}
\end{flushright}

\setlength{\baselineskip}{0.7\baselineskip}


\vfill

\begin{center}
São Paulo

Janeiro de 2023
\end{center}

\end{titlepage}
\noindent

\vspace{.07cm}
\begin{figure}[h]
\centering
	\includegraphics[scale=.7]{C:/Users/edkra/OneDrive/Imagens/eletronica/diodo.png}
	\label{ConduçãoDiodo}
	\caption{Imagem de representação do diodo}	
\end{figure}



\end{document}


