\documentclass[a4paper,12pt]{article}
\usepackage[T1]{fontenc}
\usepackage[utf8]{inputenc}
\usepackage[brazilian]{babel}
\usepackage[top = 2cm, bottom = 2cm, left = 2.5cm, right = 2.5cm]{geometry}
\linespread{1.5}
\usepackage{setspace}
\usepackage{color}
\usepackage[dvipsnames, svqnames, xllnames]{xcolor}
\usepackage{amsmath, array, amssymb}
\usepackage{graphicx}

\title{\textbf{Exercícios de Fixação - Curso de LaTeX}}
\author{Autor: Edson Rodrigues dos Santos}
\date{Data: 13 de Janeiro de 2023}

\begin{document}
\begin{tabular}{|l|c|r|}
Célula 1 & Célula 2 & Célula 3\\
Célula 4 & Célula 5 & Célula 6\\
Célula 7 & Célula 8 & Célula 9	
\end{tabular}
\hspace*{2cm}
\begin{tabular}{|l|c|r|} \hline
Célula 1 & Célula 2 & Célula 3\\ \hline
Célula 4 & Célula 5 & Célula 6\\
Célula 7 & Célula 8 & Célula 9	
\end{tabular}\\\\

\begin{tabular}{|l|c|r|} \hline
Célula 1 & Célula 2 & Célula 3\\ \hline
Célula 4 & Célula 5 & Célula 6\\ \hline
Célula 7 & Célula 8 & Célula 9\\ 	
\end{tabular}\hspace*{2cm}
\begin{tabular}{|l|c|r|} \hline
Célula 1 & Célula 2 & Célula 3\\ \hline
Célula 4 & Célula 5 & Célula 6\\ \hline
Célula 7 & Célula 8 & Célula 9\\ \hline	
\end{tabular}\\\\


\begin{tabular}{|c|c|c|c|} \hline
\multicolumn{4}{|c|}{Meses do Ano}\\ \hline
Janeiro & Fevereiro & Março & Abril\\ \hline
Maio & Junho & Julho & Agosto\\ \hline
Setembro & Outrubro & Novembro & Dezembro\\\hline	

\end{tabular}\\\\


\begin{tabular}{|c|ll|} \hline
numeros & \multicolumn{2}{|c|}{Meses do Ano e abreviação}\\\hline
1 & Janeiro & Fevereiro\\ \cline{2-3}
2 & Março & Abril\\ \cline{2-3}
3 & Maio & Junho\\ \cline{2-3} 
4 & Julho & Agosto\\ \cline{2-3}
5 & Setembro & Outubro \\ \cline{2-3} 
6 & Novembro & Dezembro\\\hline	
\end{tabular}

\begin{table}[h]
\centering
\caption{Os maiores paises do mundo em extensão } \vspace*{0.5cm}
\begin{tabular}{c |cc}
Posição & Paises & Extensão territorial ($ km^{2} $)\\
\hline
1 & Russia & 17.098.246\\
2 & Canada & 9.984.670\\
3 & China & 9.596.961\\
4 & Estados Unidos & 9.371.174\\
5 & Brasil & 8.515.767
\end{tabular}
\end{table}

\begin{figure}[h]
	\centering
	\includegraphics[scale=.6]{C:/Users/edkra/OneDrive/Imagens/Screenshots/2020-08-25.png}
	\label{ImagemTeste}
	\caption{Imagem de Teste}
\end{figure}

\begin{center}
 
	\textit{Verificou então em seu celular que 1 milha corresponde a 1.609,34 metros
Qual a distância em metros até o Hotel?
4) Uma pessoa que viaja pelos Estados Unidos suspeita que está com febre. Decide então 
comprar um termômetro e verificar. Ao fazer a medida e analisar o resultado, ela 
percebe que sua temperatura é de 96,8°F.
Qual seria a temperatura dessa pessoa na unidade do S.I?
Dica: a fórmula para converter a temperatura é 𝑇𝑐
5 =(𝑇𝐹−32)}

\end{center}

\end{document}